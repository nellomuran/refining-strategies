
In this section we introduce \SLref, which extends \SL 
with nondeterministic strategies,  an \emph{outcome quantifier} that
quantifies over possible outcomes of a strategy profile,
and more importantly, a refining operator that expresses that a
strategy refines another.
We first fix some basic notations.


\subsection{Notations}
Let $\Sigma$ be an alphabet. A \emph{finite} (resp. \emph{infinite}) \emph{word} over $\Sigma$ is an element
of $\Sigma^{*}$ (resp. $\Sigma^{\omega}$). 
The \emph{length} of a finite word $w=w_{0}w_{1}\ldots
w_{n}$ is $|w|\egdef n+1$, and $\last(w)\egdef w_{n}$ is its last
letter.
Given a finite (resp. infinite) word $w$ and $0 \leq i < |w|$  (resp. $i\in\setn$), we let $w_{i}$ be the
letter at position $i$ in $w$, $w_{\leq i}$ is the prefix of $w$ that
ends at position $i$ and $w_{\geq i}$ is the suffix that starts
at position $i$.
%If $w$ is infinite, we let $w^{i}\egdef w[i,\omega]$.
We write $w\pref w'$ if $w$ is a prefix of $w'$, and $\FPref{w}$ is
the set of finite prefixes of word $w$. 
Finally, 
%$\id$ is the identity relation on the set $X$, 
the domain of a mapping $f$ is written $\dom(f)$.

 \subsection{Syntax}
 \label{sec-SL-definition}

For convenience we fix for the rest of the paper $\APf$, a finite non-empty set of
\emph{atomic propositions}, $\Agf$, a finite non-empty set of \emph{agents} or
\emph{players}, and
$\Varf$, a finite non-empty set of \emph{variables}.



\begin{definition}%[\SLref Syntax]
  \label{def-SLi}
    The syntax of \SLref is defined by the following grammar:
    \begin{align*}
  \phi\egdef &\; p 
  \mid \neg \phi 
  \mid \phi\ou\phi 
               \mid \Estratd\phi  
               \mid \Estratnd\phi
               \mid \var \refines \var'
               \mid \bind{\var}\phi
%               \mid \unbind\phi
  \mid \Eout\psi               
      \\
      \psi\egdef &\; \phi
                   \mid \neg \psi
                   \mid \psi\ou \psi
                   \mid \X \psi
                   \mid  \psi \until \psi
    \end{align*}
     where % $p$ is an atomic proposition, $\var$ is a
 % variable, $\obs$ is an observation and $\ag$ is a player.
  $p\in\APf$, $\var\in\Varf$ and $a\in\Agf$.
\end{definition}

Formulas of type $\phi$ are called \emph{state formulas}, those of type $\psi$
are called \emph{path formulas}, and \SLref consists of all state formulas.


Temporal operators, $\X$ (next) and
 $\until$ (until), have the usual meaning. Our new \emph{refinement
   operator} expresses that the strategy denoted by a variable $\var$ is more
 restrictive than another one, or that it allows less behaviours: $\var\refines\var'$ reads as ``strategy
 $\var$ refines strategy $\var'$. The two \emph{strategy
   quantifiers}  $\Estratd$ and $\Estratnd$ quantify on deterministic
 and nondeterministic strategies, respectively: $\Estratd\phi$
 (resp. $\Estratnd\phi$) reads as ``there exists a deterministic
 (resp. nondeterministic) strategy $\var$
  such that $\phi$
 holds'', where $\var$ is a strategy variable. 
As usual, the \emph{binding operator} $\bind{\var}$ assigns a strategy to an
 agent, and $\bind{\var}\phi$ reads as ``when agent $\ag$ plays strategy $\var$,
 $\phi$ holds''.
 %The \emph{unbinding operator} $\unbind$
%  instead releases agent
% $\ag$ from her current strategy, if she has one, and
% $\bind{\unb}\phi$ reads as ``when
% agent $\ag$ is not assigned any strategy, $\phi$ holds''. 
Finally, the \emph{outcome quantifier} $\Eout$ quantifies on
   outcomes of strategies currently in use: $\Eout\psi$ reads as ``$\psi$
 holds in some
 outcome of the strategies currently used by the players''.

We use usual abbreviations $\top\egdef p\ou\neg p$, $\perp\egdef\neg\top$, $\phi\impl\phi'\egdef \neg \phi \ou \phi'$,
$\phi\equivaut\phi'\egdef \phi\impl\phi'\et \phi'\impl\phi$,
 $\F\phi \egdef \top \until \phi$,  and $\always\phi \egdef \neg \F
\neg \phi$. In addition we
define the universal variants of the two strategy quantifiers:  let
$\Astratd\phi\egdef\neg\Estratd\neg\phi$ and $\Astratnd\phi\egdef\neg\Estratnd\neg\phi$.

% %\todo{see if still necessary; if so, adapt}
For every formula $\phi\in\SLref$, we let  $\free(\phi)$ be the set of variables that appear
free in $\phi$, \ie, that
appear out of the scope of a strategy quantifier. A formula $\phi$ is a \emph{sentence} if $\free(\phi)$ is empty.
Finally, we let the \emph{size} $|\phi|$ of a formula $\phi$ be the
number of symbols in $\phi$.


\subsection{Semantics}
\label{sec-SLmodels}

The models of \SLref are the usual models of Strategy Logic, \ie concurrent game structures. 

\begin{definition}%[\CGS]
  \label{def-CGS}
  A \emph{concurrent game structure} (or
  \CGS) is a tuple
  $\CGS=(\Act,\setpos,\trans,\val,\pos_\init)$ where
   \begin{itemize}
    % \item $\APf$ is a finite non-empty set of \emph{atomic propositions},
    % \item  $\Agf$ is a finite non-empty set of \emph{players},
    \item $\Act$ is a finite non-empty set of \emph{actions},
    \item $\setpos$ is a finite non-empty set of \emph{positions},
   \item $\trans:\setpos\times \Mov^{\Agf}\to \setpos$ is a \emph{transition function}, 
  \item $\val:\setpos\to 2^{\APf}$ is a \emph{labelling function}, and
  \item $\pos_\init \in \setpos$ is an \emph{initial position}.
  \end{itemize}
\end{definition}

% We define the size $|\CGS|$ of a \CGS
% $\CGS=(\Act,\setpos,\trans,\val,\pos_\init,\obsint)$ as the size of
% an explicit encoding of the transition function: $|\CGS|\egdef
% |\setpos|\times |\Act|^{|\Agf|}\times \lceil \log(|\setpos|)\rceil$.
% We may  write $\pos\in\CGS$ for $\pos\in\setpos$.

In a position $\pos\in\setpos$, where atomic propositions $\val(\pos)$
hold, each player $\ag$ chooses an action $\mova\in\Mov$, 
and the game proceeds to position
$\trans(\pos, \jmov)$, where $\jmov\in \Mov^{\Agf}$ stands for the \emph{joint action}
$(\mova)_{\ag\in\Agf}$. Given a joint action
$\jmov=(\mova)_{\ag\in\Agf}$ and $\ag\in\Agf$, we let
$\jmov_{\ag}$ denote $\mova$.
A \emph{finite} (resp. \emph{infinite}) \emph{play} is a finite (resp. infinite)
word $\fplay=\pos_{0}\ldots \pos_{n}$ (resp. $\iplay=\pos_{0} \pos_{1}\ldots$)
such that $\pos_0=\pos_\init$ and for every $i$ such that $0\leq i<|\fplay|-1$ (resp. $i\geq 0$), there exists a joint action $\jmov$
such that $\trans(\pos_{i}, \jmov)=\pos_{i+1}$.

 \halfline
 \head{Strategies}
A (nondeterministic) \emph{strategy} is a  function $\strat:\setpos^+
\to 2^\Mov\setminus\emptyset$ that maps each finite play to a nonempty finite set of
actions that the player may play.  A strategy $\strat$ is \emph{deterministic} if for all $\fplay$,
$\strat(\fplay)$ is a singleton.
We let $\setstrat$  denote the set of all
(nondeterministic) strategies, and $\setstratd\subset\setstrat$ the
set of deterministic ones.


% \halfline
% \head{Assignments} 
An \emph{assignment}  $\assign:\Agf\union\Varf \partialto \setstrat$
is a partial function that assigns a strategy  to
each  player and strategy variable in its domain.
For an assignment
$\assign$, player $a$ and  strategy $\strat$,
$\assign[a\mapsto\strat]$ is the assignment of domain
$\dom(\assign)\union\{a\}$ that maps $a$ to $\strat$ and is equal to
$\assign$ on the rest of its domain, and 
$\assign[\var\mapsto \strat]$ is defined similarly, where $\var$ is a
variable. %  also, $\assign[a\mapsto\unb]$ is
 % the restriction of $\assign$ to domain $\dom(\assign)\setminus\{a\}$.
An assignment is
\emph{variable-complete} for a formula $\phi\in\SLref$ if
its domain contains all free variables of $\phi$.

% \halfline
% \head{Outcomes}
For an assignment $\assign$ and a finite play $\fplay$, we let
$\out(\assign,\fplay)$ be the set of infinite plays that start with
$\fplay$ and are then extended by letting players follow the strategies
assigned by $\assign$. Formally,
 $\out(\assign,\fplay)$ is the set of plays of the form $\fplay \cdot
 \pos_{1}\pos_{2}\ldots$ such that for all $i\geq 0$, there exists
 $\jmov$ such that for all $\ag\in\dom(\assign)\inter\Agf$,
 $\jmov_\ag\in\assign(\ag)(\fplay\cdot\pos_{1}\ldots\pos_i)$ \mbox{ and }
 $\pos_{i+1}=\trans(\pos_{i},\jmov)$, \mbox{ with }
 $\pos_{0}=\last(\fplay)$.
 

\begin{definition}[\SLi semantics]
\label{def-SLi-semantics}
The semantics of a state formula is defined on a \CGS $\CGS$, an
assignment  $\assign$ that is variable-complete for $\phi$, and a
finite play $\fplay$. For a path formula $\psi$, the finite play is
replaced with an infinite play $\iplay$ and an index $i\in\setn$. The
definition by mutual induction is as follows:
\[
\begin{array}{lcl}
 \CGS,\assign,\fplay\modelsSL p & \text{if} & p\in\val(\last(\fplay))\\[1pt]
 \CGS,\assign,\fplay\modelsSL \neg\phi & \text{if} &
  \CGS,\assign,\fplay\not\modelsSL\phi\\[1pt]
 \CGS,\assign,\fplay\modelsSL \phi\ou\phi' & \text{if} &
  \CGS,\assign,\fplay\modelsSL\phi \;\text{ or }\;
  \CGS,\assign,\fplay\modelsSL\phi' \\[1pt]
 \CGS,\assign,\fplay\modelsSL\Estratd\phi  & \text{if} & 
\exists\,   \strat\in\setstratd \;\text{s.t.} \;
                                                           \CGS,\assign[\var\mapsto\strat],\fplay\modelsSL \phi\\[1pt]
 \CGS,\assign,\fplay\modelsSL\Estratnd\phi  & \text{if} & 
\exists\,   \strat\in\setstratnd \;\text{s.t.} \;
    \CGS,\assign[\var\mapsto\strat],\fplay\modelsSL \phi\\[1pt]  
 \CGS,\assign,\fplay\modelsSL \bind{\var}\phi & \text{if} &
 \CGS,\assign[\ag\mapsto\assign(\var)],\fplay\modelsSL \phi\\[1pt]  
 % \CGS,\assign,\fplay\modelsSL \bind{\unb}\phi & \text{if} &
 %          \CGS,\assign[\ag\mapsto\unb],\fplay\modelsSL \phi\\[1pt]
 \CGS,\assign,\fplay\modelsSL \Eout\psi & \text{if} & \exists\iplay \in
                                                         \out(\assign,\fplay)
                                                         \text{ s.t. }
  \\
  & &\quad   \CGS,\assign,\iplay,|\fplay|-1\modelsSL \psi\\[5pt]
    \CGS,\assign,\iplay,i\modelsSL \phi & \text{if} &
                                                         \CGS,\assign,\iplay_{\leq i}\modelsSL\phi\\[1pt]
   \CGS,\assign,\iplay,i\modelsSL \neg\psi & \text{if} &
  \CGS,\assign,\iplay,i\not\modelsSL\psi\\[1pt]
 \CGS,\assign,\iplay,i\modelsSL \psi\ou\psi' & \text{if} &
  \CGS,\assign,\iplay,i\modelsSL\psi \;\text{ or }\;
  \CGS,\assign,\iplay,i\modelsSL\psi' \\[1pt]
  \CGS,\assign,\iplay,i\modelsSL\X\psi & \text{if} &
  \CGS,\assign,\iplay,i+1\modelsSL\psi\\[1pt]
\CGS,\assign,\iplay,i\modelsSL\psi\until\psi' & \text{if} & \exists\, j\geq i
   \mbox{ s.t. }\CGS,\assign,\iplay,j\modelsSL \psi' \text{ and,}\\ 
   & & \forall\, k \text{ s.t. } i\leq k <j,
\; \CGS,\assign,\iplay,k\modelsSL \psi
\end{array}
\]
\end{definition}


\todop{example}

To model check \SLref we extend the classic approach, which is to
reduce to \QCTLs, the extension of \CTLs with (second-order monadic) quantification on atomic
propositions. This logic being equivalent to MSO on infinite
trees~\cite{DBLP:journals/corr/LaroussinieM14}, it is  easy to
express that a strategy (or the atomic propositions that code for it)
refines another one.

%%% Local Variables:
%%% mode: latex
%%% TeX-master: "main"
%%% End:
