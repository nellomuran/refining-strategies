Module checking~\cite{kupferman2001module} is a generalisation of model checking to the setting
of open systems, \ie, systems that interact with an environment. The
idea is to check that the system satisfies a given property, specified
for instance in \LTL or \CTLs, for any possible environment in which
it is used.

In this setting, the environment is seen as an entity that can cut
some transitions of the system. One particular environment can thus be
modelled as a nondeterministic strategy that defines which transitions
are allowed, and the module checking problem for a \CTLs formula $\phi$ can
be written as follows in \SLref, where $E$ is the environment:

\[\phimodule:=\forall\var (E,\var) \,\phi\]

Solving the module-checking problem for \CTLs specifications can thus
be done by model checking formula $\phimodule$. This formula has
simulation depth at most 1, and the procedure thus runs  in 
doubly exponential deterministic time, which is asymptotically optimal~\cite{kupferman2001module}.

%%% Local Variables:
%%% mode: latex
%%% TeX-master: "main"
%%% End:
