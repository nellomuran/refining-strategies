% easychair.tex,v 3.5 2017/03/15

%\documentclass{easychair}

%%%% ijcai20.tex


\documentclass{article}
\pdfpagewidth=8.5in
\pdfpageheight=11in
% The file ijcai20.sty is NOT the same than previous years'
\usepackage{ijcai20}

\usepackage{hyperref}
\usepackage{newunicodechar}

\usepackage{todonotes}
\newcommand{\todop}[1]{\todo[inline,color=red!30]{\upshape TODO: #1}}

%packages

\usepackage{amsmath,amsfonts}
\usepackage{amsthm}
\usepackage{xspace}
% put these before amssymb
% to avoid ``Too many math alphabets used in version normal.''
% cf. https://tex.stackexchange.com/a/243541/91665
\newcommand{\hmmax}{0}
\newcommand{\bmmax}{0}
\usepackage{amssymb}

\usepackage{tikz}
\usetikzlibrary{automata,arrows,positioning,shapes,backgrounds,fit}
% \usepackage{doc}
% \usepackage[lofdepth,lotdepth]{subfig}

%\usepackage[toc]{appendix}

\usepackage{manfnt}
\newif\ifdraft\drafttrue
%\newif\ifdraft\draftfalse
\usepackage{draftmode}

\usepackage{enumitem}
\usepackage{bm}

% theorem environments
\theoremstyle{definition}
\newtheorem{definition}{Definition}
\newtheorem{example}{Example}
\newtheorem{remark}{Remark}
\theoremstyle{plain}
\newtheorem{theorem}{Theorem}
\newtheorem{lemma}[theorem]{Lemma}
\newtheorem{proposition}[theorem]{Proposition}
\newtheorem{corollary}[theorem]{Corollary}


\DeclareMathOperator*{\Exists}{\exists}
\DeclareMathOperator*{\Forall}{\forall}

\newcommand{\head}[1]{\textbf{#1.}}
\newcommand{\halfline}{\vskip.5em}

% Abbreviations

\newcommand{\ie}{i.e.}
\newcommand{\eg}{e.g.}
\newcommand{\egdef}{:=}
\newcommand{\wlogen}{w.l.o.g.\xspace}
\newcommand{\liste}[3]{#1_{#2},\ldots,#1_{#3}}
\newcommand{\bigiff}{\;\;\;iff\;\;\;}
%\newcommand{\st}{\;\mbox{s.t.}\;}

% General

\newcommand{\init}{{\iota}}
\newcommand{\bisim}{\leftrightarroweq}
\newcommand{\bisimrel}{R}
\newcommand{\wbisim}{\leftrightarroweq_w}
\newcommand{\wbisimrel}{R}
\DeclareMathOperator{\auxR}{\textit{R}}
\DeclareMathOperator{\auxE}{\textit{E}}
\newcommand{\dom}{\textit{dom}}
\newcommand{\codom}{\textit{codom}}
\newcommand{\partialto}{\rightharpoonup}
\newcommand{\pref}{\preccurlyeq}
\newcommand{\FPref}[1]{\textit{pref}\,(#1)}

% sets
\newcommand{\setn}{\mathbb N}     % ensemble N 
\newcommand{\inter}{\cap}
\newcommand{\union}{\cup}
\newcommand{\dunion}{\uplus}
\newcommand{\biginter}{\bigcap}
\newcommand{\bigunion}{\bigcup}
\newcommand{\sminus}{\setminus}
\newcommand{\compo}{\circ}
\newcommand{\sse}{\subseteq}

\newcommand{\id}[1][X]{\textnormal{Id}_{#1}}

%% Games

\newcommand{\setpos}{V}
\newcommand{\moves}{E}
\newcommand{\pos}{v}
\newcommand{\move}{\rightarrow}

%% Logic

% Expressivity

\newcommand{\subsumed}{\preccurlyeq}
\newcommand{\strsubsumed}{\prec}

% Propositional 

\newcommand{\AP}{\mathcal{AP}}
\newcommand{\APf}{\textnormal{AP}}
\newcommand{\pform}{\alpha}
\newcommand{\pforma}{\beta}
\newcommand{\ou}{\vee}
\newcommand{\et}{\wedge}
\newcommand{\bigou}{\bigvee}
\newcommand{\biget}{\bigwedge}
\newcommand{\impl}{\to}
\newcommand{\equivaut}{\leftrightarrow}
\newcommand{\Equivaut}{\Leftrightarrow}
\newcommand{\Longequivaut}{\Longleftrightarrow}
\renewcommand{\phi}{\varphi}
\newcommand{\depth}{d}
\newcommand{\subf}{\textit{Sub}}
\newcommand{\true}{\texttt{true}}
\newcommand{\false}{\texttt{false}}
\newcommand{\cphi}[1]{\widetilde{#1}}
\newcommand{\op}{\mbox{$\dagger$}}
\newcommand{\sem}[2]{\llbracket #1 \rrbracket^{#2}}

\newcommand{\lang}{\mathcal L}
\newcommand{\langn}[1]{\lang_{#1}}
\newcommand{\relat}{\,R\,}
\newcommand{\rel}{\leadsto}
% \newcommand{\rela}{\leadsto_1}
% \newcommand{\relb}{\leadsto_{\lga}}
\newcommand{\reli}[1][i]{\rel_{#1}}
\newcommand{\relsi}[1][i]{\rel^{\mathcal S}_{#1}}
\newcommand{\relfi}[1][i]{\rel^{\mathcal F}_{#1}}
\newcommand{\rela}[1][a]{\rel_{#1}}

% First order

\newcommand{\var}{x}
\newcommand{\varb}{y}
\newcommand{\varc}{z}
\newcommand{\setvar}{\mathcal V\textit{ar}}
\newcommand{\Var}{\setvar}
\newcommand{\Varf}{\textnormal{Var}} 
\newcommand{\free}{\textit{free}\,}
\newcommand{\model}{\mathcal M}
\newcommand{\domain}{D}
\newcommand{\interpr}{I}
\newcommand{\ass}{s}
\newcommand{\verifier}{Verifier\xspace}
\newcommand{\spoiler}{Spoiler\xspace}
\newcommand{\verif}{V}
\newcommand{\spoil}{S}
\newcommand{\genplayer}{X}
\newcommand{\egame}[3]{\ga_{#2,#3}^{#1}}


% Temporal logic

\newcommand{\Prop}{{Prop}}
\newcommand{\until}{{\bf U}}
\newcommand{\release}{{\bf R}}
\newcommand{\wuntil}{\widetilde{\bf U}}
\newcommand{\rond}{\fullmoon}
\newcommand{\X}{{\bf X}}
\newcommand{\always}{{\bf G}}
\newcommand{\F}{{\bf F}}
\newcommand{\trinf}{\pi}
\newcommand{\trfin}{\rho}
\newcommand{\E}{{\bf E}}
\newcommand{\A}{{\bf A}}

% Finite words automata

\newcommand{\w}{w}
\newcommand{\wa}{u}
\newcommand{\wb}{v}
\newcommand{\word}{\w}
%\newcommand{\pref}{\preccurlyeq}
\newcommand{\wauto}{\mathcal W}
\newcommand{\wA}{\wauto}
\newcommand{\wQ}{Q}
\newcommand{\wq}{q}
\newcommand{\wDelta}{\Delta}
\newcommand{\wdelta}{\delta}
\newcommand{\wF}{F}
\newcommand{\wFp}{F_p}
\newcommand{\wFpbar}{\overline{\wFp}}
\newcommand{\settraces}{\mathcal T}
\newcommand{\mirror}[1]{\overline{#1}}


%% Tree automata

\newcommand{\auto}{\mathcal A}
\newcommand{\nauto}{\mathcal N}
\newcommand{\ATA}{\ifmmode \auto \else ATA\xspace\fi}
\newcommand{\NTA}{\ifmmode \nauto \else NTA\xspace\fi}
\newcommand{\labeled}{labelled\xspace}
\newcommand{\labeling}{labelling\xspace}
\newcommand{\labelings}{labellings\xspace}
\newcommand{\Dirtree}{X}
\newcommand{\Dirtreea}{Y}
\newcommand{\Dirtreeb}{Z}
\newcommand{\Dirtreei}[1][I]{\setlstates_{#1}}
\newcommand{\dir}{x}
\newcommand{\dira}{y}
\newcommand{\dirz}{z}
\newcommand{\racine}{r}
\newcommand{\noeud}{u}
\newcommand{\noeuda}{v}
\newcommand{\noeudb}{w}
\newcommand{\setnodes}{U}
\newcommand{\tree}{\tau}
\newcommand{\ltree}{t}
\newcommand{\forest}{u}
\newcommand{\lforest}{\universe}
\newcommand{\arity}{\mbox{arity}}
\newcommand{\lab}{\ell}
\newcommand{\plab}[1][p]{\ell_{#1}}
\newcommand{\prodlab}{\otimes}
\newcommand{\Dir}{\mbox{\textit{Dir}}}
\newcommand{\tauto}{\mathcal A}
\newcommand{\tQ}{Q}
\newcommand{\tQp}{\tQ^\top}
\newcommand{\tQm}{\tQ^\perp}
\newcommand{\tq}{q}
\newcommand{\tDelta}{\Delta}
\newcommand{\tdelta}{\delta}
\newcommand{\couleur}{C}
\newcommand{\nword}{w}
\newcommand{\bool}{\mathbb B}
\newcommand{\boolp}{\mathbb B^+}
\newcommand{\Dalt}{\Dir_A}
\newcommand{\Dtwo}{\Dir_\uparrow}
\newcommand{\upa}{{\uparrow}}
\newcommand{\down}{{\downarrow}}
\newcommand{\stay}{\epsilon}
\protected\def\tpath{\ifmmode \lambda \else path\xspace\fi}
\newcommand{\tpaths}{paths\xspace}
\newcommand{\tPaths}{Paths}
\newcommand{\game}{\mathcal G}
\newcommand{\tgame}[3]{\mathcal G(#1,#2,#3)}
\newcommand{\proj}[2][p]{\begingroup #2\!\Downarrow_{#1}\endgroup}
\newcommand{\hide}[1][\Dirtreea]{\textnormal{hide}_{#1}}
\newcommand{\wide}[1][\Dirtreea]{\textnormal{wide}_{#1}}

%% Operations on automata

\newcommand{\cauto}{\overline{\mathcal A}}
\newcommand{\cdelta}{\overline{\delta}}
\newcommand{\ccouleur}{\overline{\couleur}}
\newcommand{\compl}[1]{\overline{#1}}
% \newcommand{\narrow}[2][Y]{\ensuremath{\textnormal{narrow}_{#1}(#2)}}
% \newcommand{\framing}[2][\tree]{\ensuremath{\textnormal{frame}_{#1}(#2)}}
\newcommand{\narrow}[2][Y]{#2\!\downarrow_{#1}}
\newcommand{\framing}[2][\tree]{\ensuremath{#2\cap #1}}

%% Concurrent game structures

\newcommand{\Ag}{\mathcal{A}\textit{g}}
\newcommand{\Agf}{\textnormal{Ag}}
\newcommand{\ag}{a}
\newcommand{\agb}{b}
\newcommand{\Act}{\textnormal{Ac}}
\newcommand{\Mov}{\Act}
\newcommand{\setmoves}{\Mov}
\newcommand{\maxmov}{l}
\newcommand{\act}{c}
\newcommand{\mov}{\act}
\newcommand{\obs}{o}
\newcommand{\mova}{\mov_\ag}
\newcommand{\jmov}{\bm{\mov}}
\newcommand{\CGS}{\relax\ifmmode\mathcal G\else$\textrm{CGS}$\xspace\fi}
\newcommand{\CGSi}{\relax\ifmmode\mathcal
  G\else$\textrm{CGS}_{\textnormal{ii}}$\xspace\fi} %_{\textnormal{ii}
\newcommand{\CGSip}{\relax\ifmmode\mathcal G'_{\textnormal{ii}}\else$\textrm{CGS}_{\textnormal{ii}}$\xspace\fi}
\newcommand{\CGSs}{$\textrm{CGS}_{\textnormal{i}}$\xspace}
\newcommand{\CGSp}{\textrm{CGS}\xspace}
\newcommand{\trans}{E}
\newcommand{\val}{\ell}
\newcommand{\fplay}{\rho}
\newcommand{\iplay}{\pi}
\newcommand{\strat}{\sigma}
\newcommand{\substrat}[2][\fplay]{{#2}_{|#1}}
\newcommand{\out}{\textnormal{Out}}
\newcommand{\play}{\out}
\newcommand{\class}{\mathcal C}
\newcommand{\obseq}[1][\obs]{\sim_{#1}}
\newcommand{\infset}[1][\ag]{I^{#1}}
% \newcommand{\FPlay}{\text{Plays}}
\newcommand{\setcontinuations}[1]{\text{Cont}(#1)}
\newcommand{\eqc}[2][\pos]{[#1]_{#2}}

\newcommand{\setstrat}{\mbox{\emph{Str}}}
\newcommand{\setstrats}[1][\pos]{\mbox{\emph{Str}}(#1)}
\newcommand{\setstrato}[1][\obs]{\mbox{\emph{Str}}_{#1}}
\newcommand{\setstratos}[2][\pos]{\mbox{\emph{Str}}_{#2}(#1)}
\newcommand{\strattrans}[2][\fplay]{#2^{#1}}
\newcommand{\assign}{\chi}
\newcommand{\passign}{\chi}
\newcommand{\assigntrans}[2][\fplay]{#2^{#1}}
\newcommand{\globaltrans}[3][\assign]{(#1,#2)^{#3}}

\newcommand{\modelsSL}{\models}%{\models_{\scriptsize{\SLi}}}
\newcommand{\modelsBSL}{\models}

%% ATL

\newcommand{\coal}{A}
\newcommand{\coala}{B}
\newcommand{\EstratATL}[1][\coal]{\langle #1 \rangle}
\newcommand{\AstratATL}[1][\coal]{[ #1 ]}


%% ATLsc

\newcommand{\Estratsc}[1][\coal]{\langle\!\cdot #1 \cdot\!\rangle}
\newcommand{\Astratsc}[1][\coal]{[ \cdot #1 \cdot ]}


%% SL operators

\newcommand{\determin}{\text{d}}
\newcommand{\nondetermin}{\text{nd}}
\newcommand{\gen}{\text{ty}}
% \newcommand{\Estratd}[1][\var]{\exists^{\hskip.5pt\determin} #1\,}
% \newcommand{\Astratd}[1][\var]{\forall^{\hskip.5pt\determin} #1\,}
\DeclareMathOperator*{\existsd}{\exists^{\hskip.5pt\determin}}
\DeclareMathOperator*{\foralld}{\forall^{\hskip.5pt\determin}}
% \newcommand{\Estratnd}[1][\var]{\exists^\nondetermin #1\,}
% \newcommand{\Astratnd}[1][\var]{\forall^\nondetermin #1\,}
% \newcommand{\Estratgen}[1][\var]{\exists^\gen #1\,}
% \newcommand{\Astratgen}[1][\var]{\forall^\gen #1\,}
\newcommand{\setstratd}{\mbox{\emph{Str}}^{\text{d}}}
\newcommand{\setstratnd}{\mbox{\emph{Str}}}
\newcommand{\refines}{\preceq}
\newcommand{\refinesc}[1][\fplay]{\preceq_{#1}}
\newcommand{\refinesstr}{\prec}
\newcommand{\refinesstrc}[1][\fplay]{\prec_{#1}}
\newcommand{\bind}[2][\ag]{(#1,#2)}
\DeclareMathOperator{\unb}{?}
\newcommand{\emptystrat}{\unb}
\newcommand{\unbind}[1][\ag]{(#1,\unb)}
\newcommand{\Aout}{{\bf A}}
\newcommand{\Eout}{{\bf E}}

\newcommand{\freeFun}[1]{\free(#1)}


\newcommand{\phidet}[1][\var]{\text{det}(#1)}
\newcommand{\maxperm}{\text{MaxPerm}}

%% Complexity

\newcommand{\qdepth}{qd}
\newcommand{\ndd}{\mbox{sd}}
\newcommand{\ndn}{\mbox{sn}}
\newcommand{\nd}{\mbox{nd}}
\newcommand{\alt}{\mbox{alt}}
\newcommand{\tower}[2]{\mathrm{exp}\big(#1 \mid #2\big)}


%% Complexity classes
\newcommand{\itexp}[2]{{\mbox{\textnormal{exp}}^{#1}(#2)}}
\newcommand{\poly}{\mbox{\textnormal{poly}}}
\protected\def\DTIME{\ifmmode \mbox{\sc Dtime} \else {\sc Dtime}\xspace\fi}
\protected\def\NTIME{\ifmmode \mbox{\sc Ntime} \else {\sc Ntime}\xspace\fi}
\protected\def\DSPACE{\ifmmode \mbox{\sc Dspace} \else {\sc Dspace}\xspace\fi}
\protected\def\NSPACE{\ifmmode \mbox{\sc Nspace} \else {\sc Nspace}\xspace\fi}
\protected\def\NP{\ifmmode \mbox{\sc NP} \else {\sc NP}\xspace\fi}
\protected\def\coNP{\ifmmode \mbox{\sc coNP} \else {\sc coNP}\xspace\fi}
\protected\def\NPSPACE{\ifmmode \mbox{\sc NPspace} \else {\sc NPspace}\xspace\fi}
\protected\def\PSPACE{\ifmmode \mbox{\sc Pspace} \else {\sc Pspace}\xspace\fi}
\protected\def\EXPSPACE{\ifmmode \mbox{\sc Expspace} \else {\sc Expspace}\xspace\fi}
\protected\def\TWOEXPSPACE{\ifmmode \mbox{\sc 2Expspace} \else {\sc 2Expspace}\xspace\fi}
\protected\def\PTIME{\ifmmode \mbox{\sc Ptime} \else {\sc Ptime}\xspace\fi}
\protected\def\NPTIME{\ifmmode \mbox{\sc NP} \else {\sc NP}\xspace\fi}
\protected\def\EXPTIME{\ifmmode \mbox{\sc Exptime} \else {\sc Exptime}\xspace\fi}
\protected\def\NEXPTIME{\ifmmode \mbox{\sc NExptime} \else {\sc NExptime}\xspace\fi}
\protected\def\2EXPTIME{\ifmmode \mbox{\sc 2-Exptime} \else {\sc
    2-Exptime}\xspace\fi}
\DeclareRobustCommand{\kEXPTIME}[1][k]{\ifmmode \mbox{\sc $#1$-Exptime}
\else {\sc $#1$-Exptime}\xspace\fi}
\DeclareRobustCommand{\kNEXPTIME}[1][k]{\ifmmode \mbox{\sc $#1$-NExptime}
\else {\sc $#1$-NExptime}\xspace\fi}
\DeclareRobustCommand{\kEXPSPACE}[1][k]{\ifmmode \mbox{\sc $#1$-Expspace}
\else {\sc $#1$-Expspace}\xspace\fi}
\protected\def\ELEMENTARY{\ifmmode \mbox{\sc Elementary} \else {\sc Elementary}\xspace\fi}

%% Names of logics

\newcommand\LTL{\textnormal{\sffamily LTL}\xspace}
\newcommand\ATL{\textnormal{{\sffamily ATL}}\xspace}
\newcommand\ATLs{\ensuremath{\textnormal{{\sffamily ATL}}^{*}}\xspace}
\newcommand\ATLssc{\ensuremath{\textnormal{{\sffamily ATL}}^{*}_{\textnormal{\scriptsize sc}}}\xspace}
\newcommand\ATLi{\ensuremath{\textnormal{{\sffamily ATL}}_{\textnormal{\scriptsize i}}}\xspace}
\newcommand\ATLsi{\ensuremath{\textnormal{{\sffamily ATL}}^{*}_{\textnormal{\scriptsize i,R}}}\xspace}
\newcommand\ATLsci{\ensuremath{\textnormal{{\sffamily ATL}}_{\textnormal{\scriptsize sc, i}}}\xspace}
\newcommand\ATLssci{\ensuremath{\textnormal{{\sffamily ATL}}^{*}_{\textnormal{\scriptsize sc,i}}}\xspace}
\newcommand\CTL{\textnormal{{\sffamily CTL}}\xspace}
\newcommand\CTLs{\ensuremath{\textnormal{{\sffamily CTL}}^{*}}\xspace}
\newcommand\QCTL{\textnormal{{\sffamily QCTL}}\xspace}
\newcommand\QCTLs{\ensuremath{\textnormal{{\sffamily
        QCTL}}^{*}}\xspace}
\newcommand\EQkCTLs{\ensuremath{\textnormal{{\sffamily EQ}}^k\textnormal{{\sffamily
        CTL}}^{*}}\xspace}
\newcommand\QCTLi{\ensuremath{\textnormal{{\sffamily QCTL}}_{\textnormal{\scriptsize i}}}\xspace}
\newcommand\QCTLsi{\ensuremath{\textnormal{{\sffamily QCTL}}^{*}_{\textnormal{\scriptsize
        ii}}}\xspace}
\newcommand\QCTLsih{\ensuremath{\textnormal{{\sffamily QCTL}}^{*}_{\textnormal{\scriptsize i,$\tiny{\subseteq}$}}}\xspace}
\newcommand\MSO{\textnormal{{\sffamily MSO}}\xspace}
\newcommand\MSOeql{\ensuremath{\textnormal{{\sffamily MSO}}_{\textnormal{eq}}}\xspace}
\newcommand\SL{\textnormal{{\sffamily SL}}\xspace}
\newcommand\SLNG{\textnormal{{\sffamily SL[\small{NG}]}}\xspace}
\newcommand\SLi{\ensuremath{\textnormal{{\sffamily
        SL}}_{\textnormal{\scriptsize ii}}}\xspace}
\newcommand\BSLi{\ensuremath{\textnormal{{\sffamily BSL}}_{\textnormal{\scriptsize ii}}}\xspace}
\newcommand{\BSL}{\text{\sffamily BSL}\xspace}
\newcommand{\BSLplus}{\ensuremath{\BSL^+}\xspace}
\newcommand{\BSLpath}{\ensuremath{\text{\sffamily BSL}_\psi}\xspace}
\newcommand{\BSLpluspath}{\ensuremath{\BSL^+_\psi}\xspace}
\newcommand\CL{\textnormal{{\sffamily CL}}\xspace}

\newcommand\SLref{\ensuremath{\textnormal{{\sffamily
        SL}}^{\!\prec}}\xspace}


%% Compound Kripke structures

\newcommand{\APlstates}{\APf_l}
\newcommand{\struct}{\mathcal S}
\newcommand{\KS}{\relax\ifmmode\struct\else\textrm{KS}\xspace\fi}
\newcommand{\KSs}{\textrm{KSs}\xspace}
\newcommand{\CKS}{\relax\ifmmode\struct\else\textrm{CKS}\xspace\fi}
\newcommand{\CKSs}{\textrm{CKSs}\xspace}
\newcommand{\setstates}{S}
\newcommand{\setlstates}{L}
\newcommand{\prodsetlstates}[1][n]{\prod_{i=1}^n\setlstates_i}
\newcommand{\sstate}{s}
\newcommand{\lstate}{l}
\newcommand{\relation}{R}
\newcommand{\spath}{\tpath}
\newcommand{\Paths}{\text{Paths}}
\newcommand{\FPaths}{\text{FinPaths}}
\newcommand{\Pequiv}[1][p]{\equiv_{#1}}
\newcommand{\last}{\mbox{last}}

\newcommand{\unfold}[2][\CKS]{\ltree_{#1}}
\newcommand{\domunfold}[2][\CKS]{\tree_{#1}(#2)}
\newcommand{\labunfold}[2][\CKS]{\lab_{#1}(#2)}

\newcommand{\cobs}{\textnormal{\textbf{o}}}
\newcommand{\oequiv}[1][\cobs]{\approx_{#1}}
\newcommand{\oequivt}[1][\cobs]{\approx_{#1}}

%% QCTLi

\newcommand{\existsp}[1][p]{\exists#1\,}
\newcommand{\forallp}[1][p]{\forall#1\,}
\newcommand{\textexistsp}[1][p]{\exists#1}
\newcommand{\modelss}{\models_s}
\newcommand{\modelst}{\models}
\newcommand{\uniq}[1][p]{\mathrm{uniq}(#1)}

%% MSO

\newcommand{\edge}{S}
\newcommand{\eql}{\text{eq}}
\newcommand{\setvarfo}{\setvar_1}
\newcommand{\setvarso}{\setvar_2}

%% Translations

\newcommand{\transs}[1]{\widetilde{#1}}
\newcommand{\transt}[1]{\widehat{#1}}
\newcommand{\ligne}[1]{\mathbf{border}(#1)}
\newcommand{\ligneb}[1]{\mathbf{level}(#1)}

%% Misc

\newcommand{\bigauto}[2][\sstate]{\ATA_{#1}^{#2}}
\newcommand{\setcobs}{\textnormal{O}}
\newcommand{\Iphi}[1][\phi]{I_{#1}}
\newcommand{\projI}[2][I]{\begingroup #2\!\downarrow_{#1}\endgroup}
\newcommand{\liftI}[3][I]{\begingroup #3\!\uparrow^{#1}_{#2}\endgroup}
\newcommand{\trees}[1][\sstate]{\tree_#1}
\newcommand{\autopower}[1][\CKS]{\wauto^{#1}}
\newcommand{\Qpower}[1][\CKS]{\wQ^{#1}}
\newcommand{\qpower}[1][\CKS]{\wq^{#1}}
\newcommand{\deltapower}[1][\CKS]{\wdelta^{#1}}
\newcommand{\autopsi}[1][\psi]{\wauto^{#1}}
\newcommand{\Qpsi}[1][\psi]{\wQ^{#1}}
\newcommand{\qpsi}[1][\psi]{\wq^{#1}}
\newcommand{\deltapsi}[1][\psi]{\wdelta^{#1}}
\newcommand{\Deltapsi}[1][\psi]{\wDelta^{#1}}
\newcommand{\couleurpsi}[1][\psi]{\couleur^{#1}}
\newcommand{\APq}{{\APf_{\exists}}}
\newcommand{\APS}{\APf_{\CKS}}
\newcommand{\APfree}{\APf_{f}}
\newcommand{\blank}{\mathbf{0}}
\newcommand{\labS}{\lab_\CKS}

\newcommand{\SI}[1][I]{\setstates_{#1}}

\newcommand{\APm}{\APf_{\mov}}
\newcommand{\APv}{\APf_{\pos}}

\newcommand{\tr}[2][f]{(#2)_s^{\,#1}} % should only be used in section V
\newcommand{\trp}[2][f]{(#2)_p^{\,#1}} % should only be used in section V

\newcommand{\trobs}[1]{\widetilde{#1}}

\newcommand{\phistrat}[1][\var]{\phi_{\text{str}}(#1)}
\newcommand{\psistrat}[1][\var]{\psi_{\text{str}}^{#1}}
\newcommand{\psistratdet}[1][\var]{\psi_{\text{str,det}}^{#1}}
\newcommand{\phistratdet}[1][\var]{\phi_{\text{str}}^{\text{det}}(#1)}
\newcommand{\phiout}[1][\assign]{\Phi_{\text{out}}^{\,#1}}
\newcommand{\psiout}[1][\assign]{\psi_{\text{out}}^{\,#1}}

\newcommand{\stratlab}[2][\sigma]{\ell_{#1}^{#2}}

%\newcommand{\coal}{A}
\newcommand{\setobs}[1][\phi]{\Obsf(#1)}

\newcommand{\gsphi}[1][\phi]{f_\CKS^{#1}}
\newcommand{\Ed}{\E\mathrm{d}}
\newcommand{\rEd}{\exists\mathrm{d}}
% \newcommand{\nEd}{\mbox{$\E$d}}
% \newcommand{\nrEd}{\mbox{$\exists$d}}

\newcommand{\Phirat}{\Phi_{\text{RS}}}
\newcommand{\Phiratc}{\Phi_{\text{c-RS}}}
\newcommand{\Phiratnc}{\Phi_{\text{nc-RS}}}
\newcommand{\Phiratci}{\Phi^{\textnormal{\scriptsize ii}}_{\text{c-RS}}}
\newcommand{\Phiratnci}{\Phi^{\textnormal{\scriptsize ii}}_{\text{nc-RS}}}
\newcommand{\vecvar}[1][\varb]{\bm{#1}}
\newcommand{\vecenv}{\bm{e}}
\newcommand{\vecbind}[2][\vecenv]{\bind[#1]{#2}}
\newcommand{\phiauxgen}[1][\vecvar]{\phi_{\gamma}}
\newcommand{\phiauxDS}[1][\vecvar]{\phi_{\text{DS}}}
\newcommand{\phiauxNE}[1][\vecvar]{\phi_{\text{NE}}}
\newcommand{\phiauxSPE}[1][\vecvar]{\phi_{\text{SPE}}}
\newcommand{\phiauxDSi}[1][\vecvar]{\phi^{\textnormal{\scriptsize ii}}_{\text{DS}}}
\newcommand{\phiauxNEi}[1][\vecvar]{\phi^{\textnormal{\scriptsize ii}}_{\text{NE}}}
\newcommand{\phiauxSPEi}[1][\vecvar]{\phi^{\textnormal{\scriptsize ii}}_{\text{SPE}}}

\newcommand{\phisynthdet}{\phi^{\determin}_{\text{synth}}}
\newcommand{\phisynthndet}{\phi^{\nondetermin}_{\text{synth}}}
\newcommand{\phisynthmp}{\phi^{\text{max}}_{\text{synth}}}

%% Coordination logic

\newcommand{\coordvar}{\mathcal C}
\newcommand{\stratvar}{\mathcal S}
\newcommand{\strquantif}{\Finv}
\newcommand{\scope}[2][\phi]{\textit{Scope}_{#1}(#2)}
\newcommand{\freecoord}[1][\phi]{\mathcal F_{#1}}
\newcommand{\APphi}[1][\phi]{\APf_{#1}}
\newcommand{\setCGS}[1][\APf]{\textnormal{\CGSi}(#1)}
\newcommand{\settrees}[2][\freecoord]{\textnormal{Trees}(#1,#2)}
\newcommand{\trmodels}{\textnormal{tr}_{1}}
\newcommand{\trformulas}{\textnormal{tr}_{2}}
\newcommand{\hidden}{\textit{hidden}}
\newcommand{\CKSfull}{\CKS_{\textit{full}}}
\newcommand{\trCL}[1]{\widetilde{#1}}

%% For \bigtimes

\DeclareFontFamily{U}{mathx}{\hyphenchar\font45}
\DeclareFontShape{U}{mathx}{m}{n}{
      <5> <6> <7> <8> <9> <10>
      <10.95> <12> <14.4> <17.28> <20.74> <24.88>
      mathx10
      }{}
\DeclareSymbolFont{mathx}{U}{mathx}{m}{n}
%\DeclareMathSymbol{\bigtimes}{1}{mathx}{"91}

%% Proof sketch

% \newenvironment{proofsketch}[1][\proofname]{\par
%   \pushQED{\qed}%
%   \normalfont \topsep6\p@\@plus6\p@\relax
%   \trivlist
%   \item[\hskip\labelsep
%         \color{darkgray}\sffamily\bfseries
%     #1\@addpunct{.}]\ignorespaces
% }{%
%   \popQED\endtrivlist%\@endpefalse
% }

\renewcommand{\iff}{\equivaut}

%% Complexity

\newcommand{\nbcol}[1]{\#({#1})}
\newcommand{\kpsi}{K_\psi}
\newcommand{\ka}{K_1}
\newcommand{\kb}{K_2}
\newcommand{\salph}[1][\phi]{2^{|\APq|}}
\newcommand{\paraphi}{H}


\title{Refining Strategies in Strategy Logic}


\author{
% Giuseppe De Giacomo$^1$\and
% Bastien Maubert$^2$\And
%  Aniello Murano$^2$\\
% 	\affiliations
% 	$^1$Sapienza Università di Roma\\
% 	$^2$Universit\`a degli Studi di Napoli Federico II\\
% 	\emails
% 	\{first, second\}@example.com,
% 	third@other.example.com,
% 	fourth@example.com
% }
%
%%  \authorrunning{} has to be set for the shorter version of the authors' names;
%% otherwise a warning will be rendered in the running heads. When processed by
%% EasyChair, this command is mandatory: a document without \authorrunning
%% will be rejected by EasyChair
%
%\authorrunning{B. Maubert, S. Pinchinat \& F. Schwarzentruber}
%
%% \titlerunning{} has to be set to either the main title or its shorter
%% version for the running heads. When processed by
%% EasyChair, this command is mandatory: a document without \titlerunning
%% will be rejected by EasyChair
%\titlerunning{Concurrent Games in Dynamic Epistemic Logic}
}
%\newcommand{\free}{*}

\begin{document}

\maketitle

\begin{abstract}
  A long lasting problem in artificial intelligence is that of
  refining strategies (or protocols, or plans): given a strategy that ensures, say, some
  property P in all behaviours that it allows (for instance a safety
  property), how to refine it to enforce, in addition to P, some
  property P' (say a liveness property)? We show that this problem can
  be solved elegantly in the framework of Strategy Logic (SL), a very
  expressive logic to reason about strategic abilities. We first
  introduce in SL nondeterministic strategies, which can be seen as
  protocols, and extend its syntax with a refinement operator. We then
  study the model-checking problem for this logic and show that the
  possibility to reason about refinement of strategies comes at no
  computational cost.
\end{abstract}

\section{Introduction}
\label{section:introduction}
This paper is about nondeterministic strategies (aka plans or protocols), i.e., strategies that associate to the current history a \emph{set of alternative moves} (instead of one) all of which are ``good'' for the objective of the strategy.
%%

Nondeterministic strategies have been studied in literature in several
context.
%%
Possibly the most relevant area is Discrete Event Control
where a central notion is that of \emph{maximally permissive
  supervisor} \cite{}. This is supervisor that controls a plant, i.e.,
allows the plant to do only certain operations at each point in
time. Note that this supervisor does not says exactly what to do to
the plant (as a deterministic strategy) but in fact tries to leave as
much freedom as possible to plant itself blocking only operations that
are unsafe. In fact it is of interest to be \emph{maximally
  permissive} wrt the plant. And indeed the central result of Discrete
Control Theory is that such a maximally permissive supervisor, i.e.,
nondeterministic strategy always exists if the plant and the
supervisor specifications are expressed as regular languages.


Another interesting case is that controllers that orchestrate several components to compose a desired global behaviror \cite{Sardina,Logan}. ONe way of seen this is that the controller tries to maintain overtime a sort of simulation relation between the desired behavior expressed as a triansition system and the cartesan product of the transitino systems of the components.


Nondeterministic strategies are of interest in several context. For example, in planning when the action precondition specification can be seen as a nonseterministi strategywe specify a function that given the state of the domain returns a set of possible actions. Now if we consider the state as summary of the relavant part of the history we can see  


Although not as common as standard deterministic strategies, they are quite common


%%% Local Variables:
%%% mode: latex
%%% TeX-master: "main"
%%% End:


\section{Strategy refinement}
\label{section:refinement}

\section{Strategy Logic with refinement}
\label{section:SL}

In this section we introduce \SLref, which extends \SL 
with nondeterministic strategies,  an \emph{outcome quantifier} that
quantifies over possible outcomes of a strategy profile,
and more importantly, a refining operator that expresses that a
strategy refines another.
We first fix some basic notations.



 \subsection{Syntax}
 \label{sec-SL-definition}

In addition to the sets of propositions $\APf$ and agents $\Agf$, we
now fix $\Varf$, a finite non-empty set of \emph{variables}.



\begin{definition}%[\SLref Syntax]
  \label{def-SLi}
    The syntax of \SLref is defined by the following grammar:
    \begin{align*}
  \phi\egdef &\; p 
  \mid \neg \phi 
  \mid \phi\ou\phi 
               \mid \Estratnd\phi
%               \mid \Estratd\phi  
               \mid \var \refines \varb
               \mid \bind{\var}\phi
%               \mid \unbind\phi
  \mid \Eout\psi               
      \\
      \psi\egdef &\; \phi
                   \mid \neg \psi
                   \mid \psi\ou \psi
                   \mid \X \psi
                   \mid  \psi \until \psi
    \end{align*}
     where 
  $p\in\APf$, $\var,\varb\in\Varf$ and $a\in\Agf$.
\end{definition}

Formulas of type $\phi$ are called \emph{state formulas}, those of type $\psi$
are called \emph{path formulas}, and \SLref consists of all state formulas.


Temporal operators, $\X$ (next) and
 $\until$ (until), have the usual meaning. The \emph{refinement
   operator} expresses that the strategy denoted by a variable $\var$ is more
 restrictive than another one, or that it allows less behaviours: $\var\refines\varb$ reads as ``strategy
 $\var$ refines strategy $\varb$''. The  \emph{strategy
   quantifier}  $\Estratnd$  has its usual meaning, except that it now
 quantifies on \emph{nondeterministic}
 strategies: $\Estratnd\phi$
 reads as ``there exists a nondeterministic
 strategy $\var$
  such that $\phi$
 holds'', where $\var$ is a strategy variable. 
As usual, the \emph{binding operator} $\bind{\var}$ assigns a strategy to an
 agent, and $\bind{\var}\phi$ reads as ``when agent $\ag$ plays strategy $\var$,
 $\phi$ holds''.
 %The \emph{unbinding operator} $\unbind$
%  instead releases agent
% $\ag$ from her current strategy, if she has one, and
% $\bind{\unb}\phi$ reads as ``when
% agent $\ag$ is not assigned any strategy, $\phi$ holds''. 
Finally, the \emph{outcome quantifier} $\Eout$ quantifies on
   outcomes of strategies currently in use: $\Eout\psi$ reads as ``$\psi$
 holds in some
 outcome of the strategies currently used by the players''.

We use usual abbreviations $\top\egdef p\ou\neg p$, $\perp\egdef\neg\top$, $\phi\impl\phi'\egdef \neg \phi \ou \phi'$,
$\phi\equivaut\phi'\egdef \phi\impl\phi'\et \phi'\impl\phi$,
 $\F\phi \egdef \top \until \phi$,   $\always\phi \egdef \neg \F
\neg \phi$ and
 $\Astratnd\phi\egdef\neg\Estratnd\neg\phi$. % and $\Astratd\phi\egdef\neg\Estratd\neg\phi$.

% %\todo{see if still necessary; if so, adapt}
For every formula $\phi\in\SLref$, we let  $\free(\phi)$ be the set of variables that appear
free in $\phi$, \ie, that
appear out of the scope of a strategy quantifier. A formula $\phi$ is a \emph{sentence} if $\free(\phi)$ is empty.
Finally, we let the \emph{size} $|\phi|$ of a formula $\phi$ be the
number of symbols in $\phi$.


\subsection{Semantics}
\label{sec-SLmodels}

 \SLref formulas are interpreted in a \CGS, and the semantics makes
 use of the following additional notions.
% \halfline

% \head{Assignments} 
An \emph{assignment}  $\assign:\Agf\union\Varf \partialto \setstrat$
is a partial function that assigns a strategy  to
each  player and strategy variable in its domain.
For an assignment
$\assign$, player $a$ and  strategy $\strat$,
$\assign[a\mapsto\strat]$ is the assignment of domain
$\dom(\assign)\union\{a\}$ that maps $a$ to $\strat$ and is equal to
$\assign$ on the rest of its domain, and 
$\assign[\var\mapsto \strat]$ is defined similarly, where $\var$ is a
variable. %  also, $\assign[a\mapsto\unb]$ is
 % the restriction of $\assign$ to domain $\dom(\assign)\setminus\{a\}$.
An assignment is
\emph{variable-complete} for a formula $\phi\in\SLref$ if
its domain contains all free variables of $\phi$.

% \halfline
% \head{Outcomes}
For an assignment $\assign$ and a finite play $\fplay$, we let
$\out(\assign,\fplay)$ be the set of infinite plays that start with
$\fplay$ and are then extended by letting players follow the strategies
assigned by $\assign$. Formally,
 $\out(\assign,\fplay)$ is the set of plays of the form $\fplay \cdot
 \pos_{1}\pos_{2}\ldots$ such that for all $i\geq 0$, there exists
 $\jmov$ such that for all $\ag\in\dom(\assign)\inter\Agf$,
 $\jmov_\ag\in\assign(\ag)(\fplay\cdot\pos_{1}\ldots\pos_i)$ \mbox{ and }
 $\pos_{i+1}=\trans(\pos_{i},\jmov)$, \mbox{ with }
 $\pos_{0}=\last(\fplay)$.
 

\begin{definition}%[\SLi semantics]
\label{def-SLi-semantics}
The semantics of a state formula is defined on a \CGS $\CGS$, an
assignment  $\assign$ that is variable-complete for $\phi$, and a
finite play $\fplay$. For a path formula $\psi$, the finite play is
replaced with an infinite play $\iplay$ and an index $i\in\setn$. The
definition by mutual induction is as follows:
\[
\begin{array}{lcl}
 \CGS,\assign,\fplay\modelsSL p & \text{if} & p\in\val(\last(\fplay))\\[3pt]
 \CGS,\assign,\fplay\modelsSL \neg\phi & \text{if} &
  \CGS,\assign,\fplay\not\modelsSL\phi\\[3pt]
 \CGS,\assign,\fplay\modelsSL \phi\ou\phi' & \text{if} &
  \CGS,\assign,\fplay\modelsSL\phi \;\text{ or }\;
  \CGS,\assign,\fplay\modelsSL\phi' \\[3pt]
 \CGS,\assign,\fplay\modelsSL\Estratnd\phi  & \text{if} & 
\exists\,   \strat\in\setstratnd \;\text{s.t.} \;
                                                          \CGS,\assign[\var\mapsto\strat],\fplay\modelsSL
                                                          \phi\\[3pt]
%    \CGS,\assign,\fplay\modelsSL\Estratd\phi  & \text{if} & 
% \exists\,   \strat\in\setstratd \;\text{s.t.} \;
%                                                            \CGS,\assign[\var\mapsto\strat],\fplay\modelsSL
%                                                          \phi\\[3pt]
   \CGS,\assign,\fplay\modelsSL \var\refines\varb & \text{if} &
                                                              \assign(\var)
                                                              \mbox{
                                                              refines
                                                                }\assign(\varb)
  \mbox{ after }\fplay\\[3pt]
 \CGS,\assign,\fplay\modelsSL \bind{\var}\phi & \text{if} &
 \CGS,\assign[\ag\mapsto\assign(\var)],\fplay\modelsSL \phi\\[3pt]  
 % \CGS,\assign,\fplay\modelsSL \bind{\unb}\phi & \text{if} &
 %          \CGS,\assign[\ag\mapsto\unb],\fplay\modelsSL \phi\\[3pt]
 \CGS,\assign,\fplay\modelsSL \Eout\psi & \text{if} & \exists\iplay \in
                                                         \out(\assign,\fplay)
                                                         \text{ s.t. }
  \\
  & &\quad   \CGS,\assign,\iplay,|\fplay|-1\modelsSL \psi\\[5pt]
    \CGS,\assign,\iplay,i\modelsSL \phi & \text{if} &
                                                         \CGS,\assign,\iplay_{\leq i}\modelsSL\phi\\[3pt]
   \CGS,\assign,\iplay,i\modelsSL \neg\psi & \text{if} &
  \CGS,\assign,\iplay,i\not\modelsSL\psi\\[3pt]
 \CGS,\assign,\iplay,i\modelsSL \psi\ou\psi' & \text{if} &
  \CGS,\assign,\iplay,i\modelsSL\psi \;\text{ or }\;
  \CGS,\assign,\iplay,i\modelsSL\psi' \\[3pt]
  \CGS,\assign,\iplay,i\modelsSL\X\psi & \text{if} &
  \CGS,\assign,\iplay,i+1\modelsSL\psi\\[3pt]
\CGS,\assign,\iplay,i\modelsSL\psi\until\psi' & \text{if} & \exists\, j\geq i
   \mbox{ s.t. }\CGS,\assign,\iplay,j\modelsSL \psi' \text{ and,}\\ 
   & & \forall\, k \text{ s.t. } i\leq k <j,
\; \CGS,\assign,\iplay,k\modelsSL \psi
\end{array}
\]
\end{definition}


We give some examples of useful notions that can be expressed in this
logic. 

\begin{example}[Strategy equality]
  First, it is easy to see that a strategy $\strat$ refines a strategy
  $\strat'$ if $\strat \refines \strat'$ and
  $\strat'\refines\strat$. We thus define the abbreviation
  \[\var = \varb \quad := \quad \var \refines \varb \wedge \varb
    \refines \var\]
  We thus have that $\CGS,\assign,\fplay\models \var = \varb$ if, and
  only if, $\substrat{\assign(\var)}=\substrat{\assign(\varb)}$. And in
  particular, $\CGS,\assign,\pos_\init\models\var=\varb$ if, and only
  if, $\assign(\var)=\assign(\varb)$.
  We also let $\var \neq \varb := \neg (\var = \varb)$ and
  $\var\refinesstr\varb := \var \refines \varb \wedge \var \neq \varb$.
\end{example}

\begin{example}[Deterministic strategies]
  We can also express that a strategy, or its refinement to
  continuations of the current finite play, is deterministic, with the
  following formula:
  \[\phidet \quad := \quad \forall \varb \; \varb \refines \var \impl \var\refines\varb\]
\end{example}

\begin{example}[Maximal permissive strategies]
  Given a formula $\phi(\var)$ we can express that a strategy $\var$
  is maximally permissive with regards to  $\phi(\var)$. Define formula $\maxperm(\var,\phi)$ as
  follows:
  \[\maxperm(\var,\phi) \quad := \quad \phi(\var) \wedge (\forall \varb \;
    \var \refinesstr \varb \impl \neg \phi(\varb))\]
  For instance, if we have two antagonistic players $\ag$ and $\agb$,
  and $\ag$ tries to ensure the safety property $\always p$, we can let $\phi(\var)=\forall \varc
  \bind[\ag]{\var}\bind[\agb]{\varc}\always p$, and it then holds that
  $\CGS,\assign,\pos_\init\models \maxperm(\var,\phi)$ if, and only
  if, $\assign(\var)$ is a maximally permissive winning strategy for $\ag$.
\end{example}

In the following sections we  show how \SLref captures a number of
important problems related to strategy synthesis and nondeterministic strategies. 






%%% Local Variables:
%%% mode: latex
%%% TeX-master: "main"
%%% End:


\section{Model checking \SLref}
\label{section:mc}
We first recall briefly the syntax and semantics of \QCTLs, to which
we will reduce \SLref.



\begin{definition}%[\QCTLs: Syntax]
  \label{def-syntax-QCTLs}
  The syntax of \QCTLs is defined by the following grammar:
  \begin{align*}
  \phi\egdef &\; p \mid \neg \phi \mid \phi\ou \phi \mid \E \psi \mid
  \existsp[p] \phi\\
    \psi\egdef &\; \phi \mid \neg \psi \mid \psi\ou \psi \mid \X \psi \mid
  \psi \until \psi
\end{align*}
where $p\in\APf$. 
\end{definition}

Again, formulas of type $\phi$ are called \emph{state formulas}, those of type $\psi$
are called \emph{path formulas}, and \QCTLs consists of all the state formulas
defined by the grammar, and   we use standard abbreviation 
$\A\psi \egdef \neg\E\neg\psi$.
% The size $|\phi|$ of a formula $\phi$ is defined inductively as usual, but the
% following case: $|\existsp{\cobs}\phi|\egdef 1 + |\cobs| + |\phi|$.

The models of \QCTLs are classic Kripke structures:

\begin{definition}
A \emph{Kripke structure}, or \KS, over $\APf$ is a tuple 
$\KS=(\setstates,\relation,\lab,\sstate_\init)$ where
\begin{itemize}
\item $\setstates$  is a set of
\emph{states}, 
\item $\relation\subseteq\setstates\times\setstates$ is a
left-total\footnote{\ie, for all $\sstate\in\setstates$, there exists $\sstate'$
such that $(\sstate,\sstate')\in\relation$.} \emph{transition
relation}, 
\item $\lab:\setstates\to 2^{\APf}$ is a \emph{\labeling function} and
\item $\sstate_\init \in \setstates$ is an \emph{initial state}.
\end{itemize}
\end{definition}

A \emph{path} in $\KS$  is an infinite sequence of states
$\spath=\sstate_{0}\sstate_{1}\ldots$ such that
 for all $i\in\setn$,
$(\sstate_{i},\sstate_{i+1})\in \relation$. 
A \emph{finite path} is a finite non-empty prefix of a path.
Similar to continuations of finite plays, given a finite path $\spath$ we write
$\setcontinuations{\spath}$ for the set of finite paths that start
with $\spath$.
We may write $\sstate\in\KS$ for $\sstate\in\setstates$, and we
define the \emph{size} $|\KS|$ of a \KS
$\KS=(\setstates,\relation,\sstate_\init,\lab)$ as its number of states: $|\KS|\egdef
|\setstates|$. 

Since we will interpret \QCTLs on unfoldings of \KS, we now define
infinite trees.

\halfline
\head{Trees}
Let $\Dirtree$ be a finite set of \emph{directions} (typically a set of states). 
An \emph{$\Dirtree$-tree} $\tree$ 
 is a nonempty set of words $\tree\subseteq \Dirtree^+$ such that
 $\bm{(1)}$ there exists $\racine\in\Dirtree$,  called the
    \emph{root} of $\tree$, such that each
    $\noeud\in\tree$ starts with $\racine$ ($\racine\pref\noeud$);
$\bm{(2)}$  if $\noeud\cdot\dir\in\tree$ and $\noeud\cdot\dir\neq\racine$, then
    $\noeud\in\tree$;
 $\bm{(3)}$ if $\noeud\in\tree$ then there exists $\dir\in\Dirtree$ such that $\noeud\cdot\dir\in\tree$.

 The elements of a tree $\tree$ are called \emph{nodes}.  
  If 
 $\noeud\cdot\dir \in \tree$, we say that $\noeud\cdot\dir$ is a \emph{child} of
 $\noeud$.
An $\Dirtree$-tree $\tree$ is \emph{complete} if for every $\noeud \in
\tree$ and  $\dir \in \Dirtree$,  $\noeud \cdot \dir \in \tree$.
A \emph{\tpath} in $\tree$ is an infinite sequence of nodes $\tpath=\noeud_0\noeud_1\ldots$
such that for all $i\in\setn$, $\noeud_{i+1}$ is a child of
$\noeud_i$,
and $\tPaths(\noeud)$ is the set of \tpaths
 that start in node $\noeud$. 

 An \emph{$\APf$-\labeled $\Dirtree$-tree}, or
\emph{$(\APf,\Dirtree)$-tree} for short, is a pair
$\ltree=(\tree,\lab)$, where $\tree$ is an $\Dirtree$-tree called the
\emph{domain} of $\ltree$ and
$\lab:\tree \rightarrow 2^{\APf}$ is a \emph{\labeling}, which maps
each node to the set of propositions that hold there.
 For $p\in\APf$, a \emph{$p$-\labeling} for a  tree is a mapping
$\plab:\tree\to \{0,1\}$ that indicates in which nodes $p$ holds, and
for a \labeled tree $\ltree=(\tree,\lab)$, the $p$-\labeling of $\ltree$ is
the $p$-\labeling $\noeud \mapsto 1$ if $p\in\lab(\noeud)$, 0 otherwise. 
The composition of a \labeled tree $\ltree=(\tree,\lab)$ with a
$p$-\labeling $\plab$ for $\tree$ is defined as
$\ltree\prodlab\plab\egdef(\tree,\lab')$, where
$\lab'(\noeud)=\lab(\noeud)\union \{p\}$ if $\plab(\noeud)=1$, and
$\lab(\noeud)\setminus \{p\}$ otherwise.
A $p$-\labeling for a labelled tree $\ltree=(\tree,\lab)$ is a
$p$-\labeling for its domain $\tree$.
A \emph{pointed labelled tree} is a pair $(\ltree,\noeud)$ where
 $\noeud$ is a node of $\ltree$.

 % \begin{definition}
 %   \label{def-unfolding}
Let $\KS=(\setstates,\relation,\lab,\sstate_\init)$ be a Kripke structure over $\APf$. 
The \emph{tree-unfolding of $\KS$} is the $(\APf,\setstates)$-tree $\unfold{\sstate}\egdef (\tree,\lab')$, where
    $\tree$ is the set
    of all finite  paths that start in $\sstate_\init$, and
    for every $\noeud\in\tree$,
    $\lab'(\noeud)\egdef \lab(\last(\noeud))$.
%  \end{definition}

\begin{definition}%[\QCTLsi semantics]
We define by induction the satisfaction relation $\modelst$ of
\QCTLs. Let   $\ltree=(\tree,\lab)$ be
an $\APf$-\labeled tree, 
$\noeud$  a node and $\tpath$  a path in $\tree$:
% \begingroup
% \addtolength{\jot}{-3pt}
\[
\begin{array}{lcl}
  \ltree,\noeud\modelst p 			& \mbox{ if } & p\in\lab(\noeud)\\[1pt]
  \ltree,\noeud\modelst \neg \phi		& \mbox{ if } & \ltree,\noeud\not\modelst \phi\\[1pt]
  \ltree,\noeud\modelst \phi \ou \phi'		& \mbox{ if } & \ltree,\noeud \modelst \phi \mbox{ or    }\ltree,\noeud\modelst \phi' \\[1pt]
  \ltree,\noeud\modelst \E\psi			& \mbox{ if } & \exists\,\tpath\in\tPaths(\noeud) \mbox{      s.t. }\ltree,\tpath\modelst \psi \\[1pt]
  \ltree,\noeud\modelst \existsp \phi & \mbox{ if }
  & \exists\,\plab \mbox{ a $p$-\labeling for
    $\ltree$ s.t.
    }\\
  & & \quad \quad  \ltree\prodlab\plab,\noeud\modelst\phi\\[1pt]
\ltree,\tpath\modelst \phi 			& \mbox{ if } & \ltree,\tpath_{0}\modelst\phi \\[1pt] 
\ltree,\tpath\modelst \neg \psi 		& \mbox{ if }
&  \ltree,\tpath\not\modelst \psi \\[1pt] 
\ltree,\tpath\modelst \psi \ou \psi'			& \mbox{ if } & \ltree,\tpath \modelst \psi \mbox{ or }\ltree,\tpath\modelst \psi' \\[1pt] 
\ltree,\tpath\modelst \X\psi 				& \mbox{ if } & \ltree,\tpath_{\geq 1}\modelst \psi \\[1pt] 
\ltree,\tpath\modelst \psi\until\psi' 		& \mbox{ if }
& \exists\, i\geq 0 \mbox{ s.t.    }\ltree,\tpath_{\geq
                                                                i}\modelst\psi' \text{ and }\\
  & & \quad \forall j \text{ s.t. }0\leq j <i,\; \ltree,\tpath_{\geq j}\modelst\psi
\end{array}
\]
%\endgroup
\end{definition}

We write $\ltree\modelst\phi$ for $\ltree,\racine\modelst\phi$,
where $\racine$ is the root of $\ltree$.     Given a \KS $\KS$  and a
\QCTLs formula $\phi$, we also write $\KS\modelst\phi$ if
$\unfold[\KS]{\sstate}\modelst\phi$.

\bam{define alternation depth}

\begin{theorem}
  \label{theo-qctls}
  The model-checking problem for \QCTLs is \kEXPTIME[(k+1)]-complete for
  formulas of alternation depth $k$.
\end{theorem}

\subsection{Reduction to \QCTLs}
\label{sec-reduction}

We use a variant of the reductions presented
in~\cite{DBLP:journals/iandc/LaroussinieM15,DBLP:conf/csl/FijalkowMMR18,BMMRV17,DBLP:conf/kr/MaubertM18,DBLP:conf/ijcai/BouyerKMMMP19},
which transform instances of the model-checking problem for various
strategic logics to (extensions of) \QCTLs.

Let $(\CGS,\Phi)$ be an instance of the \SL
model-checking problem, and assume without loss of generality that
each strategy variable is quantified at most once in $\Phi$. We define an equivalent instance of the
model-checking problem for \QCTLs.

\bam{say that the reduction preserves alternation depth}

Define the \KS $\KS_{\CGS}\egdef(\setstates,\relation,\sstate_{\init},\lab')$ where
\begin{itemize}
\item $\setstates\egdef\{\sstate_{\pos} \mid \pos\in\setpos\}$,
\item $\relation\egdef\{(\sstate_{\pos},\sstate_{\pos'})\mid
  \exists\jmov\in\Mov^{\Agf} \mbox{ s.t. }\trans(\pos,\jmov)=\pos'\}
  \subseteq \setstates^2$,
  \item $\sstate_{\init}\egdef\sstate_{\pos_{\init}}$, and
\item $\lab'(\sstate_{\pos})\egdef\val(\pos)\union \{p_{\pos}\} \subseteq \APf \cup \APv$.
\end{itemize}



For every finite play $\fplay=\pos_{0}\ldots\pos_{k}$, define
the node $\noeud_{\fplay}\egdef \sstate_{\pos_{0}}\ldots \sstate_{\pos_{k}}$ in
$\unfold[\KS_{\CGS}]{\sstate_{\pos_{0}}}$.  Note that the mapping
$\fplay\mapsto\noeud_{\fplay}$ defines a bijection between the set
of finite plays and the set of
nodes in $\unfold[\KS_{\CGS}]{\sstate_{\init}}$. % It also puts in
% bijection $\setcontinuations{\fplay}$ and $\{\noeud_\spath\mid \spath\in\setcontinuations{\}\}$


\halfline
\head{Constructing the \QCTLs formulas $\tr[f]{\phi}$}
 We now describe how to transform an \SLref formula $\phi$ and a partial
function $f:\Agf \partialto  \Varf$ into a \QCTLs
formula $\tr[f]{\phi}$ (that will also depend on $\CGS$).
Suppose that $\Mov=\{\mov_{1},\ldots,\mov_{\maxmov}\}$, and define
$\tr[f]{\phi}$ and $\trp[f]{\psi}$ by mutual induction on state and path formulas. 
The base cases are as follows:
$\tr[f]{p} 		 \egdef p$ and $\trp[f]{\phi} \egdef
\tr[f]{\phi}$. Boolean and temporal operators are simply obtained by
distributing the translation:
$\tr[f]{\neg \phi} 	 \egdef \neg \tr[f]{\phi}$, $\trp[f]{\neg
  \psi} \egdef \neg \trp[f]{\psi}$,
$\tr[f]{\phi_1\ou\phi_2}  \egdef \tr[f]{\phi_1}\ou\tr[f]{\phi_2}$,
$\trp[f]{\psi_1\ou\psi_2}  \egdef \trp[f]{\psi_1}\ou\trp[f]{\psi_2}$,
$\trp[f]{\X\psi}  \egdef \X\trp[f]{\psi}$ and $\trp[f]{\psi_1\until\psi_2}  \egdef \trp[f]{\psi_1}\until\trp[f]{\psi_2}$.


% \begin{align*}
% \tr[f]{p} 		& \egdef p  \\
% \tr[f]{\neg \phi} 	& \egdef \neg \tr[f]{\phi}\\
% \tr[f]{\phi_1\ou\phi_2} & \egdef \tr[f]{\phi_1}\ou\tr[f]{\phi_2}. \\
% \end{align*}
%where $f_i$ is defined to be $f$ restricted to domain $\freeFun{\phi_i} \cap \Agf$.
We continue with the strategy quantifier:
\[
  \begin{array}{lrl}
& \tr[f]{\Estratnd\phi}	& \egdef  \exists
                                  p_{\mov_{1}}^{\var}\ldots
                                  \exists^{\trobs{\obs}}
                                  p_{\mov_{\maxmov}}^{\var}. \phistrat
                                  \et \tr[f]{\phi}\\[5pt]
  \mbox{where} &  \phistrat & \egdef \A\always
                              \bigou_{\mov\in\Mov}p_{\mov}^{\var}
%                               \quad\mbox{ and}\\[10pt]
% & \tr[f]{\Estratd\phi}	& \egdef  \exists
%                                   p_{\mov_{1}}^{\var}\ldots
%                                   \exists^{\trobs{\obs}}
%                                   p_{\mov_{\maxmov}}^{\var}. \phistratdet
%                                   \et \tr[f]{\phi}\\[5pt]
%   \mbox{where} & \phistratdet & \egdef
% \A\always\bigou_{\mov\in\Mov}(p_{\mov}^{\var}\et\biget_{\mov'\neq\mov}\neg
% p_{\mov'}^{\var}).                              
\end{array}
\]

The intuition is that for each possible action $\mov\in\Mov$, an
existential quantification on the atomic proposition $p_{\mov}^{\var}$
``chooses'' for each  node $\noeud_{\fplay}$ of the tree 
$\unfold[\KS_{\CGS}]{\sstate_{{\pos_0}}}$ whether strategy $\var$
allows action $\mov$ in $\fplay$ or not. 
$\phistrat$  checks that at least one action is allowed in each
node, and thus that atomic propositions
$p_{\mov}^{\var}$ indeed define a (nondeterministic) strategy.
% $\phistratdet$ instead ensures that \emph{exactly one} action is chosen for strategy $\var$  in each finite
% play, and thus that atomic propositions
% $p_{\mov}^{\var}$  characterise a
% deterministic strategy.

For strategy refinement, the translation is as follows:
\[\tr[f]{\var \refines \varb} \egdef \A\always\biget_{\mov\in\Mov} p_\mov^\var
  \impl p_\mov^{\varb}.\]

Here are the remaining cases:
\[
\begin{array}{lrl}
& \tr[f]{\bind{\var}\phi}	& \egdef \tr[{f[\ag\mapsto \var]}]{\phi} \quad\quad
                          \text{for }\var\in\Varf\union\{\unb\}  \\[5pt]
\mbox{and} & \tr[f]{\Eout\psi}	& \egdef \E\,(\psiout[f] \wedge
                                 \trp[f]{\psi}), \mbox{ where}
\end{array}
\]

\begin{align*}
\psiout[f]  \egdef \always
  \bigou_{\pos\in\setpos}\Big( & p_{\pos}\, \et  \\
  & \bigou_{\jmov\in\Mov^{\Agf}} 
  (\biget_{\ag\in\dom(f)}p_{\jmov_{\ag}}^{f(\ag)}\et \X\,
                          p_{\trans(\pos,\jmov)}) \Big ).  
\end{align*}

                      
$\psiout[f]$ checks that  each player $\ag$ in the domain of $f$
follows the strategy coded by the $p_\act^{f(\ag)}$. 

   
To prove the correctness of the translation we need some additional
definitions. First, given a strategy $\strat$ and
a strategy variable $\var$ we
let  $\stratlab{\var}\egdef\{\plab[{p_\act^\var}]\mid
\act\in\Act\}$ be the family of $p_\act^\var$-\labelings for tree
$\unfold[\KS_{\CGS}]{}$ defined as follows: for each
finite play $\fplay$ and $\act\in\Act$,
we let $\plab[{p_\act^\var}](\noeud_\fplay)\egdef1$ if $\act\in\strat(\fplay)$, 0 otherwise.
For a \labeled tree $\ltree$ with same domain as
$\unfold[\KS_{\CGS}]{}$ we write $\ltree\prodlab \stratlab{\var}$ for
$\ltree\prodlab \plab[{p_{\act_1}^\var}]\prodlab\ldots\prodlab \plab[{p_{\act_\maxmov}^\var}]$.

Second, given an infinite play $\iplay$ and a point $i\in\setn$, we let
$\tpath_{\iplay,i}$ be the infinite path in
$\unfold[\KS_{\CGS}]{\sstate_{\pos_\init}}$ that starts in node
$\noeud_{\iplay_{\leq i}}$ and is defined as
$\tpath_{\iplay,i}\egdef\noeud_{\iplay_{\leq i}}\noeud_{\iplay_{\leq
    i+1}}\noeud_{\iplay_{\leq i+2}}\ldots$

Finally, we say that a partial   function $f:\Agf\partialto\Varf$ is
\emph{compatible} with an assignment $\assign$ if
 $\dom(\assign)\inter \Agf=\dom(f)$ and  for all $a \in \dom(f)$,  $\assign(a) = \assign(f(a))$.

 \begin{proposition}
   \label{prop-redux}
   For every  state subformula $\phi$ and path subformula $\psi$ of
   $\Phi$, finite play $\fplay$, infinite play $\iplay$, point
   $i\in\setn$, for every  assignment $\assign$ variable-complete for
   $\phi$ (resp. $\psi$) and
partial   function $f:\Agf\partialto\Varf$ compatible with $\assign$, assuming
   also that no $\var_i$ in $\dom(\assign)\inter \Varf=\{\var_1,\ldots,\var_k\}$ is
 quantified in $\phi$ or $\psi$, we have
   \begin{align*}
\CGS,\assign,{\fplay}\models\phi && \mbox{iff} &&
  \unfold[\KS_{\CGS}]{\sstate_{\pos_\init}}\prodlab
  \stratlab[\assign(\var_1)]{\var_1}\prodlab\ldots \prodlab
  \stratlab[\assign(\var_k)]{\var_k},\noeud_{\fplay} \modelst
                                                              \tr[f]{\phi}\\
\CGS,\assign,{\iplay},i\models\psi && \mbox{iff} &&
  \unfold[\KS_{\CGS}]{\sstate_{\pos_\init}}\prodlab
  \stratlab[\assign(\var_1)]{\var_1}\prodlab\ldots \prodlab
  \stratlab[\assign(\var_k)]{\var_k},\tpath_{\iplay,i} \modelst
  \trp[f]{\psi}     
   \end{align*}

  In addition, $\KS_{\CGS}$ is of size linear in
$|\CGS|$, and $\tr[f]{\phi}$ and $\trp[f]{\psi}$ are of size linear in $|\CGS|^2+|\phi|$.
 \end{proposition}

 \begin{proof}
   The proof is by induction on $\phi$.
We detail the case for binding,  strategy quantification, strategy refinement and
outcome quantification, the others follow simply by definition of
$\KS_{\CGS}$ for atomic propositions and induction hypothesis for
remaining cases.

\halfline
For $\phi=\var\refines\varb$,
assume that $\CGS,\assign,{\fplay}\models\var\refines\varb$.
First, observe that since $\assign$ is variable-complete for $\phi$,
$\var$ and $\varb$ are in $\dom(\assign)$.
Now we have that 
$\substrat{\assign(\var)}(\fplay')\subseteq\substrat{\assign(\varb)}(\fplay')$
for every $\fplay'\in \setcontinuations{\fplay}$. By definition of
$\stratlab[\assign(\var)]{\var}=\{\plab[{p_\act^\var}]\mid
\act\in\Act\}$ and
$\stratlab[\assign(\varb)]{\varb}=\{\plab[{p_\act^\varb}]\mid
\act\in\Act\}$, it follows that
 for each $\act\in\Act$ and  $\fplay'\in\setcontinuations{\fplay}$, if
 $\plab[{p_\act^\var}](\fplay')=1$, then
 $\plab[{p_\act^\varb}](\fplay')=1$, and thus

\[\unfold[\KS_{\CGS}]{\sstate_{\pos_\init}}\prodlab
  \stratlab[\assign(\var)]{\var}\prodlab
  \stratlab[\assign(\varb)]{\varb}\models\A\always\biget_{\mov\in\Mov} p_\mov^\var
  \impl p_\mov^{\varb}\]
Because the labellings $\stratlab[\assign(\var)]{\var}$ touch distinct
sets of atomic propositions for each variable $\var$ in
$\dom(\assign)\cap \Varf$, we can conclude this direction.

For the other direction let $\ltree=\unfold[\KS_{\CGS}]{\sstate_{\pos_\init}}\prodlab
\stratlab[\assign(\var_1)]{\var_1}\prodlab\ldots \prodlab
\stratlab[\assign(\var_k)]{\var_k}$ and assume that 
\[\ltree,\noeud_\fplay\models \A\always\biget_{\mov\in\Mov} p_\mov^\var
  \impl p_\mov^{\varb}.\]
This implies that for every $\fplay'\in\setcontinuations{\fplay}$, \[\ltree,\noeud_{\fplay'}\models\biget_{\mov\in\Mov} p_\mov^\var
  \impl p_\mov^{\varb},\] and thus $\substrat{\assign(\var)}$ refines $\substrat{\assign(\varb)}$.

\halfline
For $\phi=\bind{\var}\phi'$, we have
$\CGS,\assign,{\fplay}\models\bind{\var}\phi'$ if and only if 
$\CGS,\assign[\ag\mapsto\assign(\var)],{\fplay}\models\phi'$.
The result follows by using the induction hypothesis with assignment
$\assign[\ag\mapsto\var]$ and function
 $f[a\mapsto\var]$. This is possible because $f[a\mapsto\var]$ is compatible with $\assign[\ag\mapsto\var]$: indeed
 $\dom(\assign[\ag\mapsto\var])\inter\Agf$ is equal to
 $\dom(\assign)\inter\Agf \union \{a\}$ which, by assumption, is equal
 to $\dom(f) \union \{a\}=\dom(f[a\mapsto
 \var])$. Also
 by assumption, for all $a'\in\dom(f)$, $\assign(a')=\assign(f(a'))$, and 
by definition $\assign[a\mapsto \assign(\var)](a)=\assign(\var)=\assign(f[a\mapsto\var](a))$.


\halfline
For $\phi=\Estratnd\phi'$, assume first that
$\CGS,\assign,{\fplay}\models\Estratnd\phi'$. There exists a
nondeterministic strategy $\strat$ such that
 \[\CGS,\assign[\var\mapsto \strat],\fplay\models \phi'.\] Since $f$ is
 compatible with $\assign$, it is also compatible with assignment
 $\assign'=\assign[\var\mapsto \strat]$. By assumption, no variable in
 $\{\var_1,\ldots,\var_k\}$ is quantified in $\phi$, so that $\var\neq
 \var_i$ for all $i$, and thus $\assign'(\var_i)=\assign(\var_i)$ for
 all $i$; and because no strategy variable is
 quantified twice in a same formula,
 $\var$ is not quantified in $\phi'$, so that no variable in
 $\{\var_1,\ldots,\var_k,\var\}$ is quantified in $\phi'$.
 By induction hypothesis 
   \[\unfold[\KS_{\CGS}]{\sstate_{\pos_\init}}\prodlab
   \stratlab[\assign'(\var_1)]{\var_1}\prodlab\ldots \prodlab
   \stratlab[\assign'(\var_k)]{\var_k}\prodlab   \stratlab[\assign'(\var)]{\var},\noeud_{\fplay}
   \modelst \tr[f]{\phi'}.\]

It follows that   \[\unfold[\KS_{\CGS}]{\sstate_{\pos_\init}}\prodlab
  \stratlab[\assign'(\var_1)]{\var_1}\prodlab\ldots \prodlab
   \stratlab[\assign'(\var_k)]{\var_k},\noeud_{\fplay}
   \modelst \exists^{\trobs{\obs}} p_{\mov_{1}}^{\var}\ldots
   \exists   p_{\mov_{\maxmov}}^{\var}. \phistrat\et\tr[f]{\phi'}.\]
 Finally, since $\assign'(\var_i)=\assign(\var_i)$ for all $i$, we
 conclude that 
 \[\unfold[\KS_{\CGS}]{\sstate_{\pos_\init}}\prodlab
   \stratlab[\assign(\var_1)]{\var_1}\prodlab\ldots \prodlab
   \stratlab[\assign(\var_k)]{\var_k},\noeud_{\fplay}
   \modelst \tr[f]{\Estratnd\phi'}.\]

For the other direction, assume
that
\[\unfold[\KS_{\CGS}]{\sstate_{\pos_\init}}\prodlab
  \stratlab[\assign(\var_1)]{\var_1}\prodlab\ldots \prodlab
  \stratlab[\assign(\var_k)]{\var_k},\noeud_{\fplay} \modelst
  \tr[f]{\phi},\] and recall that
$\tr[f]{\phi}=\exists^{\trobs{\obs}} p_{\mov_{1}}^{\var}\ldots
\exists^{\trobs{\obs}}
p_{\mov_{\maxmov}}^{\var}. \phistrat\et\tr[f]{\phi'}$.  Write
$\ltree=\unfold[\KS_{\CGS}]{\sstate_{\pos_\init}}\prodlab
\stratlab[\assign(\var_1)]{\var_1}\prodlab\ldots \prodlab
\stratlab[\assign(\var_k)]{\var_k}$. There exist
 $\plab[{p_\act^\var}]$-\labelings such that
\[\ltree\prodlab  \plab[{p_{\act_1}^\var}]\prodlab\ldots\prodlab \plab[{p_{\act_\maxmov}^\var}]
  \modelst \phistrat\et\tr[f]{\phi'}.\]
By $\phistrat$, these \labelings  
 code for a strategy $\strat$. Let
 $\assign'=\assign[\var\mapsto \strat]$. For all $1\leq i\leq k$, by
 assumption $\var\neq \var_i$, and thus $\assign'(\var_i)=\assign(\var_i)$.
 The above can thus be rewritten
 \[\unfold[\KS_{\CGS}]{\sstate_{\pos_\init}}\prodlab
\stratlab[\assign'(\var_1)]{\var_1}\prodlab\ldots \prodlab
\stratlab[\assign'(\var_k)]{\var_k}\prodlab  \stratlab[\assign'(\var)]{\var}
  \modelst \phistrat\et\tr[f]{\phi'}.\]
 By induction hypothesis we have
$\CGS,\assign[\var\mapsto
\strat],\fplay\models \phi'$, hence $\CGS,\assign,\fplay\models
\Estratnd\phi'$.

\halfline
For $\phi=\Estratd\psi$, the proof is similar, using $\phistratdet$
instead of $\phistrat$.

 
\halfline
For $\phi=\Eout\psi$,
assume first that $\CGS,\assign,{\fplay}\models\E\psi$. 
There exists a play $\iplay\in\out(\assign,\fplay)$ s.t.
$\CGS,\assign,\iplay,|\fplay|-1\modelsSL \psi$. By induction
hypothesis,
\[\unfold[\KS_{\CGS}]{\sstate_{\pos_\init}}\prodlab
  \stratlab[\assign(\var_1)]{\var_1}\prodlab\ldots \prodlab
  \stratlab[\assign(\var_k)]{\var_k},\tpath_{\iplay,|\fplay|-1} \modelst
  \trp[f]{\psi}.\] Since $\iplay$ is an outcome of $\assign$, each agent $a\in\dom(\assign)\inter\Agf$ 
follows strategy $\assign(a)$ in $\iplay$.
Because  $\dom(\assign)\inter \Agf=\dom(f)$ and for all $a \in \dom(f)$,
  $\assign(a) = \assign(f(a))$, each agent $a\in\dom(f)$ follows
the  strategy $\assign(f(a))$, which is coded by atoms
$p_\mov^{f(\ag)}$ in the translation of $\Phi$. Therefore $\tpath_{\iplay,|\fplay|-1}$ also
satisfies $\psiout$, hence \[\unfold[\KS_{\CGS}]{\sstate_{\pos_\init}}\prodlab
  \stratlab[\assign(\var_1)]{\var_1}\prodlab\ldots \prodlab
  \stratlab[\assign(\var_k)]{\var_k},\tpath_{\iplay,|\fplay|-1} \modelst
  \psiout \et   \trp[f]{\psi},\] and we are done.

  For the other direction, assume that 
  \[\unfold[\KS_{\CGS}]{\sstate_{\pos_\init}}\prodlab
  \stratlab[\assign(\var_1)]{\var_1}\prodlab\ldots \prodlab
  \stratlab[\assign(\var_k)]{\var_k},\noeud_\fplay \modelst
  \E(\psiout[f] \et   \trp[f]{\psi}).\]
There exists a path $\tpath$ in $\unfold[\KS_{\CGS}]{\sstate_{\pos_\init}}\prodlab
  \stratlab[\assign(\var_1)]{\var_1}\prodlab\ldots \prodlab
  \stratlab[\assign(\var_k)]{\var_k}$ starting in
node $\noeud_\fplay$ that satisfies both $\psiout[f]$ and $\trp[f]{\psi}$.
By construction of $\KS_{\CGS}$ there exists an infinite play $\iplay$
such that $\iplay_{\leq |\fplay|-1}=\fplay$ and $\tpath=\tpath_{\iplay,|\fplay|-1}$.
By induction hypothesis, $\CGS,\assign,\iplay,|\fplay|-1 \modelsSL \psi$.
Because $\tpath_{\iplay,|\fplay|-1}$ satisfies $\psiout[f]$, $\dom(\assign)\inter \Agf=\dom(f)$, and for all $a \in \dom(f)$,
  $\assign(a) = \assign(f(a))$, it is also the case that
  $\iplay\in\out(\assign,\fplay)$, 
hence  $\CGS,\assign,\fplay \modelsSL \Eout\psi$.

% The size of $\KS_{\CGS}$, $\tr[f]{\phi}$ and $\trp[f]{\psi}$ are easily verified.

%$\tr[f]{\Estrato{\obs}\phi}$ is of size
% $O(|\CGS|^k+|\tr[f]{\phi}|)$, with $k=1$ for nondeterministic
% strategies and $k=2$ for deterministic ones.
% and it is of size $O(|\setpos|\times |\Act|^{|\Agf|}\times
% |\Agf|)\leq O(|\CGS|\times |\Agf|) \leq O(|\CGS|^2)$.
% $\tr[f]{\Eout\psi}$ is thus of size $O(|\CGS|^2+\trp[f]{\psi})$.
 \end{proof}
 
 Applying Proposition~\ref{prop-redux} to the sentence $\Phi$, $\fplay=\pos_\init$, any assignment $\assign$, and
 the empty function $\emptyset$, we get:
 \[\CGS \models \Phi \quad \mbox{if and only if}\quad
\unfold[\KS_{\CGS}]{} \models
 \tr[\emptyset]{\Phi}.\]


  
%%% Local Variables:
%%% mode: latex
%%% TeX-master: "main"
%%% End:


\bibliographystyle{named}
\bibliography{biblio}

\end{document}


%%% Local Variables:
%%% mode: latex
%%% TeX-master: t
%%% End:
